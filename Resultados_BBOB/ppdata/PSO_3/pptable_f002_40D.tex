${\bf f_{2}}$ & \multicolumn{2}{@{}c@{}}{796 \quad} & \multicolumn{2}{@{}c@{}}{797 \quad} & \multicolumn{2}{@{}c@{}}{799 \quad} & \multicolumn{2}{@{}c@{}}{799 \quad} & \multicolumn{2}{@{}c@{}}{800 \quad} & \multicolumn{2}{@{}c@{}}{802 \quad} & \multicolumn{2}{@{}c|@{}}{804} & 15 & /15\\
 & \multicolumn{2}{@{}c@{}}{$\infty$} & \multicolumn{2}{@{}c@{}}{$\infty$} & \multicolumn{2}{@{}c@{}}{$\infty$} & \multicolumn{2}{@{}c@{}}{$\infty$} & \multicolumn{2}{@{}c@{}}{$\infty$} & \multicolumn{2}{@{}c@{}}{$\infty$} & \multicolumn{2}{@{}c|@{}}{$\infty$\textit{1e4}} & 0 & /15\\